%###############################################################
\section{Related Work}\label{sec:RelatedWork}
%###############################################################
Our efforts have similarities with \emph{mechanical computers},  computers
constructed from mechanical, not electrical components. For a fascinating
nontechnical review, see \cite{McCourtney1999}.  These devices have a rich
history, from the \emph{Pascaline}, an adding machine invented in 1642 by a
nineteen-year old Blaise Pascal; Herman Hollerith's punch-card tabulator in
1890; to the mechanical devices of IBM culminating in the 1940s.  These devices
used precision gears, pulleys, or electric motors to carry out calculations.
Though our {\sc Grid-World} implementations are rather basic, 
we require none of these precision elements---merely unit-size obstacles and particles.
%Can we call these robots? Indeed, the ENIAC itself was labelled a robot by the
%associated press when it was announced in the 1940s. 

\subsection{Sliding-Block Puzzles}
Sliding-block puzzles use rectangular tiles that are constrained to move in a 2D workspace. The objective is to move one or more tiles to desired locations. They have a long history.
Hearn \cite{hearn2005complexity} and Demaine \cite{Demaine2009} showed tiles can be arranged to create logic gates, and used this technique to prove {\sc pspace} complexity for a variety of sliding-block puzzles.  Hearn expressed the idea of building computers from the sliding blocks---many of the logic gates could be connected together, and the user could propagate a signal from one gate to the next by sliding intermediate tiles.  This requires the user to know precisely which sequence of gates to enable/disable.  In contrast to such a hands-on approach, with our architecture we can build circuits, store parameters in memory, and then actuate the entire system in parallel using a global control signal.

\subsection{Other Related Work on Programmable Matter}
Clearly there is a wide range of interesting scenarios for developing approaches to programmable matter.
One such model is the \emph{abstract Tile-Assembly Model} (aTAM) by Winfree~\cite{Winf98,WLWS98,LaWiRe99}, which has 
sparked a wide range of theoretical and practical research. In this model, unit-sized pixels (``tiles'')
interact and bond with the help of differently labeled edges, eventually composing complex assemblies.
Even though the operations and final objectives in this model are quite different from our particle computation with global
inputs (e.g., key features of the aTAM are that tiles can have a wide range of different edge types, and
that they keep sticking together after bonding), there is
a remarkable geometric parallelism to a key result of our present paper:
While it is widely believed that at the most basic level of interaction (called {\em temperature 1}),
computational universality {\em cannot be achieved}~\cite{LSAT1,ManuchTemp1,IUNeedsCoop} in the aTAM with only unit-sized pixels, 
very recent work~\cite{fhp+-ucapt-15} shows that computational universality {\em can be achieved} as soon as even slightly bigger tiles are used. 
This resembles the results of our paper, which shows that unit-size particles are insufficient for universal computation, while employing bigger particles suffices

