



%###############################################################
\section{Analysis}\label{sec:Analysis}
%###############################################################
This section analyzes the time and space required for a factory and gives simulation results.


%###############################################################
\subsection{Running Time}\label{sec:runningTime}
%###############################################################
Running a factory simulation has three phases, ramp up, production, and wind down.
During the $n-1$ \emph{ramp up}  cycles, the first polyomino is being constructed one tile at a time and no polyominos are produced.
Clever design of delays in the part hoppers ensures no unconnected tiles are released.
During \emph{production} cycles, one complete polyomino is produced each cycle.
Once the first part hopper empties, the $m-1$ \emph{wind down}  cycles each produce a complete polynomino as successive hoppers also empty.
 This section analyzes running time, defined as the time required each commanded move until all tiles are blocked and all unblocked tiles move unit distance in unit time.
 There are two results, the \emph{construction time}, the time required to assemble a single polynomino, and
 the \emph{cycle time}, the time required during production cycles to produce a single polyomino and advance all partial assemblies one cycle.
 Since a polyomino contains $n$ tiles, the \emph{construction time} $= n \cdot$ \emph{cycle time}.
 
 \todo{tell us the big-Oh for  \emph{cycle time}.  Is it $n$ or $n^2$?}



%###############################################################
\subsection{Space Required}\label{sec:requiredSpace}
%###############################################################
The space required by a factory is a function of the size of individual sub assemblies.

 \todo{tell us the big-Oh for  \emph{r} $\cdot$ \emph{c}.  Is it $n^2$ or $n^3$?}


%###############################################################
\subsection{Simulation Results}\label{sec:simResults}
%###############################################################

Algorithms  \ref{alg:FindBuildPath}, \ref{alg:BuildFactory}, and \ref{alg:FactoryAddTile}  were coded in {\sc Matlab} and are available at \cite{Manzoor2017gitAssemply}.  

This section examines experiments that generate actual running time and space required for different parts. As shown in Fig.~\ref{fig:timeplot}, the  \todo{what do we learn?}.

As shown in Fig.~\ref{fig:sizeplot}, the required size is  upper bounded by column-shaped polynominos and lower bounded by row-shaped polyominos, and is $O(n^3)$ ????.


\begin{figure}
   \centering
\begin{overpic}[width =1\columnwidth]{timeplot.pdf}
\end{overpic}
\caption{\label{fig:timeplot}Running time plotted against number of tiles (N).  
\textcolor{red}{ add max and min bounds, show results for column and row parts and arbitrary parts}
}
\end{figure}


\begin{figure}
   \centering
\begin{overpic}[width =1\columnwidth]{Factorysizeplot.pdf}
\end{overpic}
\caption{\label{fig:sizeplot}
Factory size grows quadratically (???) with the number of tiles, and is upper bounded by column-shaped polynominos and lower bounded by row-shaped polyominos.
%Sheryl -- this caption is obviousFactory size (rows$\cdot$columns) plotted against number of tiles ($n$).  
%
}
\end{figure}


