%###############################################################
\section{Experiment}\label{sec:Experiment}
%###############################################################

\subsection{Macro-scale, Gravity-Based Prototype}



\begin{figure}
   \centering
\begin{overpic}[width =\columnwidth]{MacroScalePrototype}
\end{overpic}
\caption{\label{fig:24tilefactory}A large-scale demonstration of particle assembly using gravity as the external force and magnetic attraction between red and blue particles for assembly.
}
\end{figure}


\subsection{Milli-scale, Magnetic-Based Prototype}
To demonstrate the algorithm, we developed a magnetic control stage and alginate micro-particles.

\paragraph{Experimental setup}


\begin{figure}
   \centering
\begin{overpic}[width =\columnwidth]{BastExp1.pdf}
\end{overpic}
\caption{\label{fig:Magneticstage}Experimental platform.  %\todo{describe the system}
}
\end{figure}

This stage generates a magnetic drag force by moving a permanent magnet. The permanent magnet is able to move $x, y$ direction as following two mail shafts. The permanent magnet has  T and the dimension is cm$^2$. The main channel is made up PDMS and it was filled with motility buffer. The alginate microrobot was fabricated using ~~~. After the alginate microrobots were located at each chamber in the channel, the experimental channel was located on the center of the stage where a magnet was positioned initially. The stage controller was manipulated by a C++ programming through an Arduino UNO. The channel was observed by a stereo microscope and the installed camera captured all sequent images (fps). The scheme of system is shown in Fig.~\ref{fig:Magneticstage}.

\paragraph{Experimental result}
Using one of construction maps, it is available to demonstrate the map using multiple alginate microrobots. The initial scene is shown in Fig.~\ref{fig:Construction}a and the first assemble was manipulated moving the magnet in a clockwise direction as indicated in Fig.~\ref{fig:Construction}b. The alginate microrobots moved in the oriented direction until coming into contact with an object. The final completion of a square polyomino is shown in the lower right corner in Fig.~\ref{fig:Construction}c. In addition,  other polyominoes were simultaneously being manufactured. 


\begin{figure}
   \centering
\begin{overpic}[width =\columnwidth]{BastExp2.pdf}
\end{overpic}
\caption{\label{fig:Construction}Fig. Construction of a microrobotic polyomino from four alginate artificial cells. (a) Initial position of alginate microrobots at all chambers, (b) First assemble by two microrobots from two chambers, (c) Final result of construction.
}
\end{figure}