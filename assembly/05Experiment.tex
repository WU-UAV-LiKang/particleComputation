%###############################################################
\section{Experiment}\label{sec:Experiment}
%###############################################################

%\subsection{Macro-scale, Gravity-Based Prototype}
%
%To demonstrate Algorithms 1-5 
%
%
%\begin{figure}
%   \centering
%\begin{overpic}[width =\columnwidth]{MacroScalePrototype}
%\end{overpic}
%\caption{\label{fig:24tilefactory}A large-scale demonstration of particle assembly using gravity as the external force and magnetic attraction between red and blue particles for assembly.
%}
%\end{figure}
%
%
%\subsection{Milli-scale, Magnetic-Based Prototype}
To demonstrate the algorithm, we developed a magnetic control stage and alginate micro-particles.

\paragraph{Experimental setup}


\begin{figure}
   \centering
\begin{overpic}[width =\columnwidth]{BastExp1.pdf}
\end{overpic}
\caption{\label{fig:Magneticstage}Experimental platform.  %\todo{describe the system}
}
\end{figure}


We designed a custom magnetic control stage to generate the global control inputs. 
This stage generates a magnetic drag force by moving a permanent magnet. 
This permanent magnet can translate in $x$ and $y$-axes, actuated by stepper motors and moving on linear rails. 
The neodymium permanent magnet field strength is 433 mT and dimensions are 25.4$\times$25.4 mm$^2$. 
The assembly workspace is laser cut into PDMS to a depth of   XXXX mm and then filled with motility buffer. 
All microrobots used for these experiments are loaded alginate paramagnetic hydrogels, otherwise known as artificial cells. 
The alginate microrobots were fabricated using a centrifugal method, as described in previous work \cite{ ali2016fabrication}.
 The average microrobot size is 300 $\mu$m, and were composed of a concentration of 5\% (w/v) Alginate-Na and 5\% (w/v) concentration of CaCl$_2$, and then impregnated with nano-paramagnetic particles. 


After the alginate microrobots were loaded at each hopper in the microfluidic factory layout, the experimental channel was placed at the center of the stage. 
Next, the magnet centered beneath the microfluidic factory layout. 
This position was saved as the home position for the permanent magnet. 
Stepper motors controlled the stage position. 
An Arduino UNO programmed in C++ commanded these stepper motors using a 2Hz control loop. 
After a command was initiated, such as each direction in the $ \langle u,r,d,l \rangle$ sequence, the permanent magnet was returned to the home position to better control the distribution of the magnetic gradient.  
The layout was observed through a stereomicroscope and the installed camera (Motion Pro X3) captured the procedure at 30 fps. Using ??? magnification in the stereomicroscope, the observed field of view is 23.6 $\times$ 18.9 mm$^2$.
  A system schematic is shown in Fig.~\ref{fig:Magneticstage}. 


\paragraph{Experimental result}
Using a factory layout generated by Alg.~\ref{alg:BuildFactory}, we demonstrated micro-scale assembly using multiple alginate microrobots. The initial scene is shown in Fig.~\ref{fig:Construction}(a) and the first assembly operation was orchestrated by moving the magnet in a clockwise direction, following the $ \langle u,r,d,l \rangle$  sequence as indicated in Fig.~\ref{fig:Construction}(b). Each input was applied sufficiently long to ensure all alginate microrobots moved until they touched a wall. The completed square polyomino is shown in the lower right corner in Fig.~\ref{fig:Construction}(c). In addition, other square polyominoes were manufactured simultaneously.  


\begin{figure}
   \centering
\begin{overpic}[width =\columnwidth]{BastExp2.pdf}
\end{overpic}
\caption{\label{fig:Construction}
Construction of a microrobotic polyomino from four alginate artificial cells. (a) Initial position of alginate microrobots in all chambers, (b) First assembly by two microrobots from two chambers, (c) Final result of construction. The scale bar represents XXX $\mu$m.
}
\end{figure}